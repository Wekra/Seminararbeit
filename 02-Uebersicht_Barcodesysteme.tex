\section{"Ubersicht verschiedener Barcodesysteme}
Zur Zeit befinden sich viele verschiedene Barcodesysteme in Benutzung. Manche sind spezifisch für einen bestimmten Wirtschaftszweig, andere wiederum findet man so gut wie überall. Dieser Abschnitt soll die bekanntesten und verbreitetsten Systeme vorstellen. Dabei wird eine Grobeinteilung anhand der Anzahl der Dimensionen des Codes vorgenommen.

\subsection{Eindimensionale Barcodes}
Eindimensionale Barcodes codieren Daten entlang ihrer Länge (also links und rechts); nicht ihrer Höhe (also oben und unten). Würde man nun eine Linie durch einen eindimensionalen Barcode ziehen, findet man diese Dimension. Dieses Vorgehen nutzen die meisten der verwendeten Laserscanner. Dass eindimensionale Barcodes meist sehr groß abgedruckt sind hat einen einfachen Grund: Je größer der Code, desto einfacher findet ihn ein Scanner.

\subsubsection{Strichcode}
\begin{wrapfigure}[6]{r}{0.45\textwidth}
	\centering
	\vspace{-0.2cm}
\includegraphics[width=0.25\textwidth]{Bilder/Barcode.png}
	\vspace{-0.2cm}
	\caption[]{Strichcode\footnotemark}
	\label{barcode}
\end{wrapfigure}
\footnotetext[1]{Quelle: \url{https://upload.wikimedia.org/wikipedia/commons/thumb/3/3e/Codabar-example.svg/2000px-Codabar-example.svg.png}, zuletzt abgerufen am 05.12.2016}
%TODO
Für den Strichcode gibt es seit 1993 vier europäische Standards, die ... : ISO 15420, ISO 15417, ISO 16388 und ISO 16390.
Der Strichcode wird vielfältig eingesetzt und heißt je nach Einsatzgebiet auch anders. Im Handel kennt man ihn unter der Bezeichnung EAN (European Article Number), in den Bereichen Materialfluss, Logistik und Lager als EAN128.

\subsection{Zweidimensionale Barcodes}
Zweidimensionale Barcodes enthalten Informationen entlang ihrer X- und ihrer Y-Achse. Anders gesagt, horizontal werden andere Daten codiert als vertikal.
Damit der Code richtig entschlüsselt werden kann, muss ein Scanner beide Dimensionen simultan, das Symbol also als Ganzes erfassen. Dies ist möglich durch verschieben der Scanlinie eines Laserscanners oder durch verwenden eines Laserscannerarrays welches als Kamera fungiert. Ebenso ist der Einsatz einer Kamera möglich, die den Code anschließend durch bildverarbeitende Algorithmen entschlüsselt.

\subsubsection{Stapelcode}
Stapelcodes bestehen aus mehreren Lagen von linearen Barcodes, die vertikal übereinander angeordnet wurden.
Der spezielle Typ 'Composite Bar Code' besteht aus einem linearen und einem zweidimensionalen Part.

\subsubsection{Matrixcodes}
Matrixcodes codieren Daten durch benutzen von Zellen innerhalb einer Matrix (daher auch der Name). Diese benutzten Zellen werden für die Darstellung innerhalb des Symbols schwarz gefärbt. Jede Zelle innerhalb des Codes ist gleich groß. 
Vorteile der Matrixcodes: Es werden viele Daten auf kleinem Raum gespeichert, die Lesbarkeit ist hoch, der Code kann auch bei schlechter Druckqualität oder Beschädigung noch gelesen werden und unterstützt den vollen ASCII Zeichensatz.
Nachteil: Es benötigt eher eine Kamera als eine Laserscanner, um den Code zu lesen. 
\samepage
\subsubsection{QR-Code}
\begin{wrapfigure}[7]{l}{0.35\textwidth}
	\centering
	\vspace{-0.65cm}
	\includegraphics[width=0.2\textwidth]{Bilder/QR_Code.png}
	\vspace{-0.2cm}
	\caption[]{QR-Code\footnotemark}
	\label{qrcode}
\end{wrapfigure}
\footnotetext[2]{Quelle: \url{https://upload.wikimedia.org/wikipedia/commons/thumb/9/9b/Wikipedia_mobile_en.svg/2000px-Wikipedia_mobile_en.svg.png}, zuletzt abgerufen am 05.12.2016}
Lorem ipsum dolor sit amet, consetetur sadipscing elitr, sed diam nonumy eirmod tempor invidunt ut labore et dolore magna aliquyam erat, sed diam voluptua. At vero eos et accusam et justo duo dolores et ea rebum. Stet clita kasd gubergren, no sea takimata sanctus est Lorem ipsum dolor sit amet. Lorem ipsum dolor sit amet, consetetur sadipscing elitr, sed diam nonumy eirmod tempor invidunt ut labore et dolore magna aliquyam erat, sed diam voluptua. At vero eos et accusam et justo duo dolores et ea rebum. Stet clita kasd gubergren, no sea takimata sanctus est Lorem ipsum dolor sit amet.
~\pagebreak
