\section{Geschichte des Barcodes}
Der Barcode oder auch Strichcode ist älter als man glaubt. Bereits im Jahr 1949 wurden von Norman Joseph Woodland und Bernard Silver erste Experimente mit dem Lesen von Strichcodes durchgeführt. Drei Jahre später, 1952, bekamen die beiden dann ein Patent für ihre Erfindung. 
Bevor der Barcode allerdings in den Supermarkt gelangte, verging noch einige Zeit mit experimentieren und verbessern. Erst 1973 wurde in den USA der sogenannte Universal Product Code (UPC) eingeführt. Ein weiteres Jahr später wurde dann das weltweit erste Produkt an einer Supermarktkasse gescannt: eine Packung Kaugummi.\\
1977 wurde in Europa ein eigener Barcode eingeführt, der European Article Number Code, kurz EAN-Code. 
In der Anfangszeit der Barcodes wurden diese sehr häufig fehlerhaft oder gar nicht gelesen, da bislang noch eine geeignete Drucktechnik fehlte und somit jeder Barcode eine andere Größe hatte. Heutzutage sind die Barcodes sehr genau spezifiziert und ausgereift. Auch die Drucker und Scanner haben sich enorm weiterentwickelt, sodass es heute kaum noch zu Lesefehlern kommt.\\
Wirft man einen genaueren Blick auf Europa, so fällt auf, dass es neben dem EAN-Code noch eine Unzahl verschiedener Subtypen des Barcodes gibt, etwa um einzelne Produktgruppen zu trennen. So ist zum Beispiel ein 13-stelliger EAN-Code für Bücher spezifiziert worden: die International Standard Book Number, auch als ISBN bekannt.
Speziell in Deutschland sind auch 8-stellige EAN-Codes zu finden genauso wie individuelle Barcodes, welche zum Beispiel auf Quittungen in der Bewirtung zu finden sind.\\
Wenn man nun denkt, die Entwicklung der Barcodes wäre nicht mehr wirklich vorangeschritten da sich die abgedruckten Barcodes auf den Waren der Supermärkte augenscheinlich nicht geändert haben, der ist auf dem Holzweg. \\
Eine Unterart des Barcodes, der QR-Code, wurde bereits 1994 in Japan entwickelt, hat in den letzten Jahren seinen Erfolgszug erst dort und schließlich auch weltweit angetreten und ist immer häufiger auf Plakatwänden, Postern, in Zeitungen und überall sonst zu finden\cite{Plotz2015}. Auf ihn soll in dieser Seminararbeit ein Schwerpunkt gelegt werden.

\subsection{An- und Verwendung von Barcodes}
Die Automatisierung in Lagern beziehungsweise generell in der Industrie forderte recht schnell geeignete Identifikationssysteme, welche im Stande sind, den Material- und Informationsfluss miteinander zu verknüpfen. Die Barcodes (auch Balken- oder Strichcode genannt) ermöglichen es auf recht einfachem Weg, die gedruckten Daten der Codes maschinell zu lesen während das Material oder Produkt auf einem Fließband am Scanner vorbeifährt.\\
Heute werden Barcodes in fast allen Bereichen der Industrie, des Handels, der Behörden und auch generell des täglichen Lebens eingesetzt. Sie sind auf Kassen- und Lagerzetteln, Lieferscheinen, Ausweisen, Produktetiketten und auf vielem weiteren zu finden.
\pagebreak
